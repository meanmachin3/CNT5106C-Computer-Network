%%%%%%%%%%%%%%%%%%%%%%%%%%%%%%%%%%%%%%%%%
% Structured General Purpose Assignment
% LaTeX Template
%
% This template has been downloaded from:
% http://www.latextemplates.com
%
% Original author:
%  Ted Pavlic (http://www.tedpavlic.com)
% Modified by:
%  Joe Del Rocco (https://joe.delrocco.org)
%%%%%%%%%%%%%%%%%%%%%%%%%%%%%%%%%%%%%%%%%

%----------------------------------------------------------------------------------------
%  PACKAGES AND CONFIGURATION
%----------------------------------------------------------------------------------------

\documentclass[fleqn]{article}
\usepackage{geometry}
\usepackage{fancyhdr} % For custom headers
\usepackage{lastpage} % To determine the last page for the footer
\usepackage{extramarks} % For headers and footers
\usepackage[most]{tcolorbox} % For problem answer sections
\usepackage{graphicx} % For inserting images
\usepackage{xcolor} % For link coloring
\usepackage[hidelinks]{hyperref} % For URL links (no box or color name)
\usepackage{bm}
\usepackage{amsmath}
\usepackage{amssymb}
\usepackage[english]{babel}
\usepackage[utf8x]{inputenc}
\usepackage{amsmath}
\usepackage{tikz}
\usetikzlibrary{arrows,automata}

% Margins
\geometry{
a4paper,
tmargin=1in,
bmargin=1in,
lmargin=1in,
rmargin=1in,
textwidth=6.5in,
textheight=9.0in,
headsep=0.25in
}

% Header and footer
\pagestyle{fancy}
\lhead{\myName} % Top left header
\chead{\myCourse: \myAssignment} % Top center header
\rhead{\firstxmark} % Top right header
\lfoot{\lastxmark} % Bottom left footer
\cfoot{} % Bottom center footer
\rfoot{Page\ \thepage\ of\ \pageref{LastPage}} % Bottom right footer
\renewcommand\headrulewidth{0.4pt} % Size of the header rule
\renewcommand\footrulewidth{0.4pt} % Size of the footer rule

% Other configurations
\setlength\parindent{0pt} % Removes all indentation from paragraphs
\setlength\parskip{1pt} % Ensures paragraphs are still recognizable as such
\setcounter{secnumdepth}{0} % Removes default section numbers
\setcounter{tocdepth}{3} % Sets depth of table of contents
\linespread{1.1}

% Template values
% \newcommand{\myLogo}{starfleet.jpg}
\newcommand{\myName}{Manish Yadav}
\newcommand{\myJobTitle}{3836-6483}
\newcommand{\myCompany}{Starfleet Academy}
\newcommand{\myLocation}{1701 Lincoln Blvd, San Francisco, CA}
\newcommand{\myURL}{www.starfleet.edu}
\newcommand{\myEmail}{m.yadav@ufl.edu}
\newcommand{\myCourse}{CNT5106C}
\newcommand{\mySection}{Fall 2020}
\newcommand{\myTeacher}{Dr. Ye Xia}
\newcommand{\myAssignment}{Homework 3}
\newcommand{\myDueDate}{Fri,\ Oct\ 9,\ 2020}
\newcommand{\norm}[1]{\left\lVert#1\right\rVert}


%----------------------------------------------------------------------------------------
%  DOCUMENT STRUCTURE (MACROS & ENVIRONMENTS)
%----------------------------------------------------------------------------------------

% Colored links macro
\newcommand{\hrefcol}[3] {\href{#1}{\textcolor{#3}{#2}}}

% Creates a counter to keep track of the number of problems
\newcounter{homeworkProblemCounter}

% Macro for custom title page signature header
\newsavebox{\myTitleSignature}
\sbox{\myTitleSignature}{%
\begin{tabular*}{\textwidth}{@{}l@{}@{\extracolsep{0.125in}}l@{}}%
\parbox[c][]{2.5in}{{\textbf{\myName} \par}
                    {\small \myJobTitle \par}
                    {\small \hrefcol{mailto:\myEmail}{\myEmail}{blue}} \par}
\end{tabular*}}

% Header and footer for when a page split occurs within a problem environment
\newcommand{\enterProblemHeader}[1]{%
\nobreak\extramarks{#1}{#1 continued on next page\ldots}\nobreak%
\nobreak\extramarks{#1 (continued)}{#1 continued on next page\ldots}\nobreak%
}

% Header and footer for when a page split occurs between problem environments
\newcommand{\exitProblemHeader}[1]{%
\nobreak\extramarks{#1 (continued)}{#1 continued on next page\ldots}\nobreak%
\nobreak\extramarks{#1}{}\nobreak%
}

\newcommand{\homeworkProblemName}{} % Argument = name of problem; default = "Problem #"
\newenvironment{homeworkProblem}[1][Problem \arabic{homeworkProblemCounter}]{%
\stepcounter{homeworkProblemCounter}% % Increase counter for number of problems
\renewcommand{\homeworkProblemName}{#1}% % Assign \homeworkProblemName the argument
\section{\homeworkProblemName}% % Make a section in the document with the custom problem count
\enterProblemHeader{\homeworkProblemName}% % Header and footer within environment
}{%
\exitProblemHeader{\homeworkProblemName}% % Header and footer after environment
}

\newcommand{\problemAnswer}[1]{ % Defines the problem answer command with the content as the only argument
\begin{tcolorbox}[breakable,enhanced,colback=gray!5!white,title=Answer]%
#1
\end{tcolorbox}%
% Alternative - Makes the box around the problem answer and puts the content inside
%\noindent\framebox[\columnwidth][c]{\begin{minipage}{0.98\columnwidth}#1\end{minipage}}
}

\newcommand{\homeworkSectionName}{}
\newenvironment{homeworkSection}[1]{% % For sections w/in problems; Argument = name of section (no default)
\renewcommand{\homeworkSectionName}{#1}% % Assign \homeworkSectionName the argument
\subsection{\homeworkSectionName}% % Make a subsection with the name of the subsection
\enterProblemHeader{\homeworkProblemName\ [\homeworkSectionName]}% % Header and footer within environment
}{%
\enterProblemHeader{\homeworkProblemName}% % Header and footer after environment
}

%----------------------------------------------------------------------------------------
%   TITLE PAGE
%----------------------------------------------------------------------------------------
\begin{document}

% Blank out the traditional title page
\title{\vspace{-1in}} % no title name
\author{} % no author name
\date{} % no date listed
\maketitle % makes this a title page

% Use custom title macro instead
\usebox{\myTitleSignature}
\vspace{1in} % spacing below title header

% Assignment title
{\centering \huge \myAssignment \par}
{\centering \noindent\rule{4in}{0.1pt} \par}
\vspace{0.05in}
{\centering \myCourse~: \mySection~: \myTeacher \par}
{\centering Due \myDueDate \par}
%{\centering Prepared w/ \LaTeX \par}
\vspace{1in}

% Table of Contents
\tableofcontents
\newpage

%----------------------------------------------------------------------------------------
%	PROBLEM 1
%----------------------------------------------------------------------------------------

%\begin{homeworkProblem}[Exercise \#\arabic{homeworkProblemCounter}] % Use for custom section title
\begin{homeworkProblem}
\begin{homeworkSection}{P5}
Suppose that the UDP receiver computes the Internet checksum for the received UDP segment and finds that it matches the value carried in the checksum field. Can the receiver be absolutely certain that no bit errors have occurred? Explain. \\
\problemAnswer{
    No. The receiver cannot be certain that no bit errors have occurred. \\
    
    It may be possible that both data and checksum have been corrupted that could result in checksum matched at the receiving end \cite{stone2000crc}. \\
    
    Also, 2-bit error changes in different word (i.e $1 \rightarrow 0$ and $0 \rightarrow 1$) will not change the sum. Because checksum is merely an addition. Consider two bytes (trimmed down example of WORD) $0000 0101$ and $0000 0011$, this would result in a sum of $0000 1000$. But if 1 bit from each of the bytes is flipped, i.e $0000 0111$ and $0000 0001$ would result in a sum of $0000 1000$ which is the same. \\
    
    It could also fail to detect if there is a transposition of the word i.e. $0011$ and $0101$ is being transmitted instead of $0101$ and $0011$
    
}
\end{homeworkSection}
\end{homeworkProblem}
%----------------------------------------------------------------------------------------
\pagebreak

\begin{homeworkProblem}
\begin{homeworkSection}{P6}
Consider our motivation for correcting protocol \texttt{rdt2.1}. Show that the receiver, shown in \textbf{Figure 3.57}, when operating with the sender shown in \textbf{Figure 3.11}, can lead the sender and receiver to enter into a deadlock state, where each is waiting for an event that will never occur \\
\problemAnswer{
    The real problem lies with the fact that the sender has to receive an acknowledgment from the receiver in order to send the next packet. \\
    
    To best illustrate this scenario, consider a scenario where the receiver received a packet, sent an ACK, and now has moved to wait for another packet 0. However, the ACK sent is corrupted and the sender sends back packet 1 again. Now, both the sender and receiver have entered a state where they keep sending packet 1 and the receiver keeps sending NACK since it is now waiting for packet 0 to arrive.
}
\end{homeworkSection}
\end{homeworkProblem}
%----------------------------------------------------------------------------------------
\pagebreak

\begin{homeworkProblem}
\begin{homeworkSection}{P12}
The sender side of \texttt{rdt3.0} simply ignores (that is, takes no action on) all received packets that are either in error or have the wrong value in the \texttt{acknum} field of an acknowledgment packet. Suppose that in such circumstances, \texttt{rdt3.0} were simply to retransmit the current data packet. Would the protocol still work? (Hint: Consider what would happen if there were only bit errors; there are no packet losses but premature timeouts can occur. Consider how many times the $n^{th}$ packet is sent, in the limit as $n$ approaches infinity.) \\
\problemAnswer{
    Yes. The protocol would still work. There would be packet re-transmission in case if the ACK was completely lost or had errors.\\
    
    Consider a case where packet 1 was sent. In case of premature timeout, an extra copy would be sent that would be ACK'd by the receiver. This would cause the sender to send an extra copy of packet 2. This extra copy would be ACK'd along with the previous one causing packet 3 to be sent twice. This process continues to repeat on and on. Since $n$ approaches infinity, the number of the packet sent would tend towards infinity as well.
}
\end{homeworkSection}
\end{homeworkProblem}
%----------------------------------------------------------------------------------------
\pagebreak




\bibliographystyle{plain}
\bibliography{references} % see references.bib for bibliography management
\end{document}
%----------------------------------------------------------------------------------------
%	DONE
%----------------------------------------------------------------------------------------
